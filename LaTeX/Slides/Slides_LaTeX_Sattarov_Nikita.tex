\documentclass[pdf, 10pt, unicode]{beamer}
\setbeamertemplate{page number in head/foot}[totalframenumber]

\usetheme[numbers,totalnumbers,compress, nologo]{Statmod}
\usefonttheme[onlymath]{serif}
\setbeamertemplate{navigation symbols}{}

\usepackage[T2A]{fontenc}
\usepackage[utf8]{inputenc}
\usepackage[russian]{babel}

\usepackage{graphicx}
\usepackage[style=authoryear-ibid,backend=biber]{biblatex}
\usepackage{amssymb}
\usepackage{amsthm}

\usepackage{lipsum}



\usepackage{beamerthemesplit}
\setbeamerfont{institute}{size=\normalsize}
\setbeamercovered{transparent}

\newcommand{\ds}{\displaystyle}

\title{Применение цепей Маркова в анализе данных}

\author{ Саттаров Никита Дмитриевич, группа 19.Б04 }


\institute{
	Санкт-Петербургский Государственный Университет \\
	Математико-механический факультет \\
	Кафедра статистического моделирования \\
	
	\vspace{0.5cm}
	
	Научный руководитель~--- к.ф.-м.н. Н.Э. Голяндина\\
}

 \date{
	Санкт-Петербург\\
	2021г.
}

\begin{document}
	\maketitle
	
	
	
	
	\begin{frame}\frametitle{Введение}
		Имеем таблицу данных:
		\begin{table}[h]
			\begin{center}
				\begin{tabular}{|c|c|c|}
					\hline
					Номер наблюдения & 		Начальное состояние		& Конечное состояние \\ \hline
					1 & $s_{start_{1}}$ & $s_{finish_{1}}$ \\ \hline
					2 & $s_{start_{2}}$ & $s_{finish_{2}}$ \\ \hline
					3 & $s_{start_{3}}$ & $s_{finish_{3}}$ \\ \hline
					... & ... & ... 
				\end{tabular}
			\end{center}
		\end{table}
	$\forall i: \ s_{start_{i}}, s_{finish_{i}} \in S, \ |S| < +\infty$
	
	Хотим извлечь из таблицы информацию: вероятность попасть из $s_{i}$ в $s_{j}$, среднее количество шагов, чтобы попасть из $s_{i}$ в $s_{j}$.
	\end{frame}
	
	\begin{frame}\frametitle{Постановка задачи}
		Последовательность дискретных случайных величин ${\ds \{X_{n}\}_{n \geq 0} }$ называется цепью Маркова, если
		\begin{equation*}
			\mathbb{P}(X_{n+1}=i_{n+1} \mid X_{n} = i_{n}, \ldots , X_{0} = i_{0}) = \mathbb{P} (X_{n+1}=i_{n+1} \mid X_{n} = i_{n})
		\end{equation*}
			Хотим проверить корректность построения цепи Маркова по таблице данных.
			 \begin{itemize}
			 	\item Построим цепь Маркова, которая будет соответствовать вероятностям попасть из состояния $s_i$ в состояние $s_j$ для нашей таблицы данных
				\item Сгенерируем новые наблюдения на основе вероятностей из цепи Маркова
				\item Построим новую цепь Маркова по сгенерированным наблюдениям
				\item Сверим две получившиеся цепи
		 	\end{itemize}

	\end{frame}
	
	
	\begin{frame}\frametitle{Обыкновенный метод Монте-Карло}

			 $(S_i)_{i \in \mathbb{N}}$~--- последовательность случайных строк из $\mathbb{S}^{(n)}$
			\begin{itemize}
			 \item Для любого $N$ имеем несмещённую оценку $p$:
			\begin{equation*}
				\hat{p}_{N} = \frac{\#\{S_i : d(S_0, S_i) > a\}}{N}
			\end{equation*} 
			
		\item	 Дисперсия оценки: 
			\begin{equation*}
				\mathbb{D}\hat{p}_N = \frac{p(1-p)}{N}
			\end{equation*}
			
		\item	 Относительная ошибка: 
			\begin{equation*}
				\frac{\sqrt{\mathbb{D}\hat{p}_{N}}}{p} = \sqrt{\frac{1-p}{Np}}
			\end{equation*}
			
		\end{itemize}	 
		\end{frame}

%%%%%%%%%%%%%%%%%%%%%%%%%%%%%55

		\begin{frame} \frametitle{Обыкновенный метод Монте-Карло}

		    
		\begin{itemize}
		
		\itemДля относительной ошибки $\varepsilon$ получаем требуемый порядок объема выборки:
			\begin{equation*}
				N\sim \frac{1-p}{\varepsilon^2p}
			\end{equation*}
        \item Тогда, напрмер, для оценивания некоторой вероятности $p_0\sim 10^{-10}$ с относительной ошибкой $\varepsilon = 0.01$ имеем необходимый объем выборки $N\sim 10^{17}$.
        
        \item Актуальна задача уменьшения дисперсии при фиксированном объеме выборки. 
        \end{itemize}

	\end{frame}
%%%%%%%%%%%%%%%%%%%%%%%%%%%%%%%%%%%%%%%%
	\begin{frame}\frametitle{AMS}
		\textbf{{\large Адаптивное многоуровневое расщепление}}~\parencite{Brehier2014}
		\begin{itemize}
			\item Введем $J$ промежуточных уровней: $a_0 = 0 < a_1 < \ldots < a_J = a$ и соответствующие им вероятности: 
			\begin{equation*}
				p_j = \mathbb{P}(X > a_j | X > a_{j-1})
			\end{equation*}
			\item Малая вероятность $p$ должна удовлетворять равенству:
			\begin{equation*}
				p = \prod\limits_{i = 1}^{J}p_i
			\end{equation*}

		\item Дисперсия будет минимизирована, если $p_1 = \ldots = p_J = p^{1/J}$, поэтому промежуточные уровни определяются так, чтобы выполнялось равенство множителей $p_j$. 
		\end{itemize}
	\end{frame}
	
	\begin{frame}\frametitle{AMS}

	    \noindentПри попытке применить алгоритм возникли следующие трудности: 
	    \begin{itemize}
			\item Необходимо получать условные распределения $\mathcal{L}(X|X > a_i)$. 
			\item Алгоритм определен и доказан для с.в. $X \in \mathbb{R}$, но рассматриваем $d(S_0, S_1) \in \mathbb{Z}^{+}$. 
			
			Поэтому в текущем виде применить его для оценивания малых вероятнстей, связанных со строками, нельзя. Требуется адаптация алгоритма. 
		\end{itemize}

	\end{frame}
	


	
\end{document}