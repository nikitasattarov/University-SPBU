\documentclass{article}
%\usepackage{xcolor}
\usepackage[linesnumbered,ruled,vlined]{algorithm2e}

\title{Algorithm template with Function}
\author{Roy}

%%% Coloring the comment as blue
%\newcommand\mycommfont[1]{\footnotesize\ttfamily\textcolor{blue}{#1}}
\SetCommentSty{mycommfont}




\begin{document}
\maketitle

\begin{algorithm}[H]

  \DontPrintSemicolon

  \SetKwFunction{FCalc}{$calculate\_definite\_integral\_of\_f$}

  \SetKwProg{Fn}{Def}{:}{}
  \SetKwProg{FWhile}{While}{:}{}
  \SetKwProg{FIf}{If}{:}{}
  \SetKwProg{FElse}{Else}{:}{}
  \Fn{\FCalc{$f$, $initial\_step\_size$}}{
        ''''''\;
        This algorithm calculates the definite integral of a function from 0 to 1, adaptevily, by choosing smaller steps near problematic points.\;
        ''''''\;
        x = 0.0\;
        h = initial\_step\_size\;
        accumulator = 0.0\;
      	\FWhile{{$x < 1.0$}}{
        	\FIf{{$x + h > 1.0$}}{
        		h = 1.0 - x    $\#$At end of unit interval, adjust last step to end at 1.\;
        	}
        	\FIf{{error\_too\_big\_in\_quadrature\_of\_f\_over\_range(f, [x, x + h])}}{
        		h = make\_h\_smaller(h)\;
		}
		\FElse{{$ $}}{
			accumulator += quadrature\_of\_f\_over\_range(f, [x, x + h])\;
			x += h\;
			\FIf{{error\_too\_small\_in\_quadrature\_of\_over\_range(f, [x, x + h])}}{
				h = make\_h\_larger(h) $\#$Avoid wasting time on tiny steps.
			}
		}
	}
        \KwRet accumulator\;
  }
\end{algorithm}

\end{document}