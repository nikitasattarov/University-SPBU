\documentclass{article}
\usepackage[utf8]{inputenc}
\usepackage[english, russian]{babel}
\usepackage{wrapfig}
\usepackage{amsmath}
\usepackage{amsfonts}
\usepackage{graphicx}
\usepackage{ dsfont }
\usepackage[a4paper, top = 1.5cm, bottom=1.5cm,left=2cm,right=1cm]{geometry}
\usepackage{tikz}

\pagenumbering{gobble}


\title{University. Probability Theory. Tasks and solutions.}
\author{Nikita Sattarov}
\date{Updated: \today}

\begin{document}
\maketitle
\newcommand{\ds}{\displaystyle}

\newcommand*\circled[1]{\tikz[baseline=(char.base)]{\node[shape=circle,draw,inner sep=2pt] (char) {#1};}}


\newcommand{\answer}{\\ \textbf{Ответ: }}
\newcommand{\solution}{\\ \textbf{Решение: }}

\renewcommand{\theenumi}{\arabic{enumi}}
\renewcommand{\labelenumi}{\circled{\theenumi}}

\newcommand{\task}[3]{\item {#1} \\ \solution{#2} \\ \answer {#3}.}



\noindent

\section*{Комбинаторика}
\begin{enumerate}
	\item Колода в $52$ карты раскладывается в $2$ ряда, лежащие друг под другом. Найти вероятность того, что:
        \begin{itemize}
            \item все красные карты~--- друг под другом ($A_1$); 
            \item все пики во 2-ом ряду лежат под королями ($A_2$).
        \end{itemize}
        
        \solution{}
        
        Количество всех возможных перестановок из $52$ карт: $card(\Omega) = 52!$.

        Первая подзадача:
        
        Выбираем $13$ красных карт, которые будут лежать сверху. Выбираем места для этих $13$ карт. Переставляем красные карты сверху. Переставляем красные карты снизу. Переставляем все оставшиеся карты. Получаем количество всех возможных удовлетворяющих перестановок: $card(A_1) = C_{26}^{13} \cdot C_{26}^{13} \cdot 13! \cdot 13! \cdot 26! = \left( C_{26}^{13} \right) ^ 2 \cdot \left( 13! \right) ^ 2 \cdot 26!$.
        
        Таким образом: $\ds{P(A_1) = \frac{card(A_1)}{card(\Omega)} = \frac{\left( C_{26}^{13} \right) ^ 2 \cdot \left( 13! \right) ^ 2 \cdot 26!}{52!}}$.

        Вторая подзадача:

        Рассмотрим три случая (три суммы):
        \begin{itemize}
            \item[1.] Король Пик оказывается в нижнем ряду.

            Тогда мы выбираем $i$ пиковых карт, выбираем для всех нижних пиковых карт (включая короля) $(i + 1)$ место. Переставляем карты. Далее выбираем $(i + 1)$ из оставшихся трёх королей. Переставляем их по закрепленным местам. Далее выбираем места для оставшихся пик сверху. Переставляем оставшиеся пики. далее переставляем все оставшиеся карты. Просуммируем по $i$ от $0$ до $2$.

            Получаем количество всех возможных удовлетворяющих перестановок в первом случае:
            
            $card(A_2^1) = \sum\limits_{k = 0}^{2} C_{12}^k \cdot C_{26}^{k + 1} \cdot (k+1)! \cdot C_3^{k + 1} \cdot (k+1)! \cdot C_{25 - k}^{12 - k} \cdot (12 - k)! \cdot (38 - k)!$.

            \item[2.] Король Пик оказывается в верхнем ряду и под ним лежит не пик.

            Тогда выбираем $i$ пиковых карт, выбираем места под них. Переставляем. Выбираем из трёх оставшихся $i$ королей. Переставляем по фиксированным местам. Далее выбираем места для оставшихся пик сверху. Переставляем оставшиеся пики. Далее переставляем все оставшиеся карты. Просуммируем по $i$ от $0$ до $3$.

            Получаем количество всех возможных удовлетворяющих перестановок во втором случае:

            $card(A_2^2) = \sum\limits_{k = 0}^{3} C_{12}^k \cdot C_{26}^{k} \cdot k! \cdot C_3^{k} \cdot k! \cdot C_{26 - k}^{12 - k} \cdot (12 - k)! \cdot (39 - k)!$.

            \item[3.] Король Пик оказывается в верхнем ряду и под ним лежит пик.

            Тогда выбираем $i$ пиковых карт (включая ту, которая лежит под Королем Пик). Выбираем места под них. Переставляем. Выбираем $(i−1)$ из $3$ оставшихся королей (потому что один Король~--- это Король Пик). Переставляем $i$ королей (уже включая Короля Пик). Далее выбираем места для оставшихся пик сверху. Переставляем оставшиеся пики. Переставляем оставшиеся карты. Просуммируем по $i$ от $1$ до $4$.

            Получаем количество всех возможных удовлетворяющих перестановок в третьем случае:

            $card(A_2^3) = \sum\limits_{k = 1}^{4} C_{12}^k \cdot C_{26}^{k} \cdot k! \cdot C_3^{k - 1} \cdot k! \cdot C_{26 - k}^{12 - k} \cdot (12 - k)! \cdot (40 - k)!$.

            Таким образом:

            $\ds{P(A_2) = \frac{card(A_2)}{card(\Omega)}}$.
        \end{itemize}

        Суммируем количество всех возможных удовлетворяющих перестановок в трех случаях: 

       $card(A_2) = & \sum\limits_{k = 0}^{2} C_{12}^k \cdot C_{26}^{k + 1} \cdot (k+1)! \cdot C_3^{k + 1} \cdot (k+1)! \cdot C_{25 - k}^{12 - k} \cdot (12 - k)! \cdot (38 - k)! + \\ + & \sum\limits_{k = 0}^{3} C_{12}^k \cdot C_{26}^{k} \cdot k! \cdot C_3^{k} \cdot k! \cdot C_{26 - k}^{12 - k} \cdot (12 - k)! \cdot (39 - k)! + \\ + & \sum\limits_{k = 1}^{4} C_{12}^k \cdot C_{26}^{k} \cdot k! \cdot C_3^{k - 1} \cdot k! \cdot C_{26 - k}^{12 - k} \cdot (12 - k)! \cdot (40 - k)!$.
        
        \answer{}
        
        \begin{itemize}
            \item $\ds{\frac{\left( C_{26}^{13} \right) ^ 2 \cdot \left( 13! \right) ^ 2 \cdot 26!}{52!}}$
            \item $\ds{\bigg(& \sum\limits_{k = 0}^{2} C_{12}^k \cdot C_{26}^{k + 1} \cdot (k+1)! \cdot C_3^{k + 1} \cdot (k+1)! \cdot C_{25 - k}^{12 - k} \cdot (12 - k)! \cdot (38 - k)! \; +} \\ \ds{+ & \sum\limits_{k = 0}^{3} C_{12}^k \cdot C_{26}^{k} \cdot k! \cdot C_3^{k} \cdot k! \cdot C_{26 - k}^{12 - k} \cdot (12 - k)! \cdot (39 - k)! \; }+ \\ \ds{+ & \sum\limits_{k = 1}^{4} C_{12}^k \cdot C_{26}^{k} \cdot k! \cdot C_3^{k - 1} \cdot k! \cdot C_{26 - k}^{12 - k} \cdot (12 - k)! \cdot (40 - k)!\bigg) \; / \; 52!}$.
        \end{itemize}
        
    \item Из колоды $52$ карты достают $5$, а потом из оставшихся карт~--- еще $5$. Найти вероятность того, что:
    \begin{itemize}
        \item число пик и треф во второй кучке совпадает ($A_1$);
        \item в первой кучке есть пики, а во второй~--- нет ($A_2$);
        \item число треф в первой кучке такое же, как число пик во второй кучке ($A_3$).
    \end{itemize}

    \solution{}
    
    Количество всех возможных выборов $5$ карта их кучки, а затем ещё $5$ карт из кучки: 
    $card(\Omega) = C_{52}^{5} \cdot C_{47}^{5}$.

    Первая подзадача:

    Нам нет разницы, в какой кучке будет равное количество пик и треф. Поэтому поменяем кучки местами в условии задачи. Выберем $k$ пик и $k$ треф в первую кучку. Доберём карты других мастей в первую кучку. И выберем любые $5$ карт во вторую кучку. Просуммируем по $k$ от $0$ до $2$, так как больше пик и треф не влезет. Получаем количество всех возможных удовлетворяющих перестановок: $card(A_1) = \sum\limits_{k = 0}^{2} (C_{13}^k) ^ 2 \cdot C_{26}^{5 - 2k}\cdot C_{47}^5$.

    Таким образом: $\ds{P(A_1) = \frac{card(A_1)}{card(\Omega)} = \frac{\sum\limits_{k = 0}^{2} (C_{13}^k) ^ 2 \cdot C_{26}^{5 - 2k}\cdot C_{47}^5}{C_{52}^{5} \cdot C_{47}^{5}}}$.

    Вторая подзадача:

    Выберем $k$ пиковых карт для первой кучки. Доберём оставшиеся карты для первой кучки. Доберём непиковых карт для второй кучки. Просуммируем по $k$ от $1$ до $5$. Получаем количество всех возможных удовлетворяющих перестановок: $card(A_2) = \sum\limits_{k = 1}^{5} C_{13}^k \cdot C_{39}^{5 - k} \cdot C_{34 + k}^{5}$.

    Таким образом: $\ds{P(A_2) = \frac{card(A_2)}{card(\Omega)} = \frac{\sum\limits_{k = 1}^{5} C_{13}^k \cdot C_{39}^{5 - k} \cdot C_{34 + k}^{5}}{C_{52}^{5} \cdot C_{47}^{5}}}$.

    Третья подзадача:

    Пусть $k$~--- количество выпавших треф в первой кучке, $m$~--- количество выпавших пик в первой кучке. Тогда мы сначала выбираем пики и трефы в первую кучку. Добираем оставшиеся карты в первую кучку. Далее выбираем пики во вторую кучку. Добираем оставшиеся карты во вторую кучку $(52 − 13 − k − (5 − k − m))$. Суммируем по $m$ от $0$ до $(5 − k)$, так как пик и треф в сумме не может быть больше $5$. И далее суммируем по $k$ от $0$ до $5$. Получаем количество всех возможных удовлетворяющих перестановок: 
    
    $card(A_3) = \sum\limits_{k = 0}^{5} \sum\limits_{m = 0}^{5 - k} C_{13}^{k} \cdot C_{13}^m \cdot C_{26}^{5 - k - m} \cdot C_{13 - m}^{k} \cdot C_{34 + m}^{5 - k}$.

    Таким образом: $\ds{P(A_3) = \frac{card(A_3)}{card(\Omega)} = \frac{\sum\limits_{k = 0}^{5} \sum\limits_{m = 0}^{5 - k} C_{13}^{k} \cdot C_{13}^m \cdot C_{26}^{5 - k - m} \cdot C_{13 - m}^{k} \cdot C_{34 + m}^{5 - k}}{C_{52}^{5} \cdot C_{47}^{5}}}$.

    \answer

        \begin{itemize}
            \item $\ds{\frac{\sum\limits_{k = 0}^{2} (C_{13}^k) ^ 2 \cdot C_{26}^{5 - 2k}\cdot C_{47}^5}{C_{52}^{5} \cdot C_{47}^{5}}}$
            \item $\ds{ \frac{\sum\limits_{k = 1}^{5} C_{13}^k \cdot C_{39}^{5 - k} \cdot C_{34 + k}^{5}}{C_{52}^{5} \cdot C_{47}^{5}}}$
            \item $\ds{ \frac{\sum\limits_{k = 0}^{5} \sum\limits_{m = 0}^{5 - k} C_{13}^{k} \cdot C_{13}^m \cdot C_{26}^{5 - k - m} \cdot C_{13 - m}^{k} \cdot C_{34 + m}^{5 - k}}{C_{52}^{5} \cdot C_{47}^{5}} }$.
        \end{itemize}

    \item $n$ раз бросают игральную кость. Найдите вероятности следующих событий:
    \begin{itemize}
        \item самое маленькое из выпавших чисел равно $3$ ($A_1$);
        \item двойка выпадет три раза, а тройка~--- два раза ($A_2$);
        \item появится $k_1$ единиц, $k_2$ двоек, \ldots, $k_6$ шестерок $(k1 + k2 + \ldots + k6 = n)$ ($A_3$);
        \item первая двойка появится раньше, чем первая шестерка ($A_4$).
    \end{itemize}

    \solution{}

    Всего вариантов выпавших кубиков среди $n$ бросков равно: $card(\Omega) = 6 ^ n$.

    Первая подзадача:

    Расписываем вероятность через вероятность минимумов: $P(A_1) = P(min < 3) - P(min < 4) = (1 - P(min \geq 3)) - (1 - P(min \geq 4)) = P(min \geq 4) - P(min \geq 3) = \left(\frac{4}{6}\right)^n - \left(\frac{3}{6}\right) ^ n$.

    Вторая подзадача:

    Выбираем $3$ места для двойки, затем $2$ места для тройки. Затем доставляем оставшиеся карты: $card(A_2) = C_n^3 \cdot C_{n - 3}^2 \cdot 4 ^ {n - 5}$.

    Таким образом: $\ds{P(A_2) = \frac{card(A_2)}{card(\Omega)} = \frac{C_n^3 \cdot C_{n - 3}^2 \cdot 4 ^ {n - 5}}{6 ^ n}}$.

    Третья подзадача:

    Выбираем $k_1$ мест для единицы, $k_2$ мест для двойки и так далее:
    
    $card(A_3) = C_n{k_1} \cdot C_{n - k_1}^{k_2} \cdot C_{n - k_1 - k_2}^{k_3} \cdot C_{n - k_1 - k_2 - k_3}^{k_4} \cdot C_{k_5 + k_6}^{k_5} \cdot C_{k_6}^{k_6}$.

    Таким образом: $\ds{P(A_3) = \frac{card(A_3)}{card(\Omega)} = \frac{C_n{k_1} \cdot C_{n - k_1}^{k_2} \cdot C_{n - k_1 - k_2}^{k_3} \cdot C_{n - k_1 - k_2 - k_3}^{k_4} \cdot C_{k_5 + k_6}^{k_5}}{6 ^ n}}$.

    Четвертая подзадача:

    Вероятность того, что двойка выпадет до шестёрки, равна вероятности того, что шестёрка выпадет до двойки. Для добора до полной вероятности не хватает случая, когда двойка и шестёрка вообще не выпали. Полная вероятность равна $1$. Вычитаем вероятность, когда двойка и шестёрка не выпали, и делим пополам: $\ds{P(A_4) = \frac{1}{2} \left(1 - \left(\frac{4}{6}\right) ^ n \right)}$.

    \answer{}

    \begin{itemize}
        \item $\ds{\left(\frac{4}{6}\right)^n - \left(\frac{3}{6}\right) ^ n}$
        \item $\ds{ \frac{C_n^3 \cdot C_{n - 3}^2 \cdot 4 ^ {n - 5}}{6 ^ n}}$
        \item $\ds{\frac{C_n{k_1} \cdot C_{n - k_1}^{k_2} \cdot C_{n - k_1 - k_2}^{k_3} \cdot C_{n - k_1 - k_2 - k_3}^{k_4} \cdot C_{k_5 + k_6}^{k_5}}{6 ^ n}}$
        \item .$\ds{\frac{1}{2} \left(1 - \left(\frac{4}{6}\right) ^ n \right)}$.
    \end{itemize}

    \item Путь начинается из точки $(0, 0)$. За каждый шаг человек продвигается с вероятностью $\frac{1}{2}$ направо и с вероятностью $\frac{1}{2}$ вверх на расстояние $1$. Чему равна вероятность того, что:
    \begin{itemize}
        \item он пройдет через точку с координатами $(m, n)$ ($A_1$);
        \item он пройдет через точку с координатами $(m, n)$, при этом побывав в точке с координатами $(k, l)$ $(1 \leq k < m, 1 \leq l < n)$ ($A_2$).
    \end{itemize}

    \solution{}

    Общее количество путей из точки $(0, 0)$ в точку $(m, n)$ равно: $card(\Omega) = 2^{m + n}$.

    Первая подзадача:

    Представляем шаги в виде последовательностей $0$ (шаг вправо) и $1$ (шаг наверх). В этой последовательности должно быть $(m = n)$ значений. При этом ровно $m$ нулей и $n$ единиц. Применяем формулу перестановок с повторениями: \ds{P(A_1) = \left( \frac{(m + n)!}{m! \cdot n!}\right) / \left( 2 ^ {m + n}\right)}.

    Вторая подзадача:

    То же самое. Только две последовательности. Сначала $k$ нулей и $l$ единиц. Затем $(m - k)$ нулей и $(n - l)$ единиц: $\ds{P(A_2) = \left( \frac{(k + l)!}{k! \cdot l!} \cdot \frac{((m+n) - (k + l))!}{(m - k)! \cdot (n - l)!}\right) / \left( 2 ^ {m + n}\right)}$.

    \answer{}

    \begin{itemize}
        \item $\ds{\left( \frac{(m + n)!}{m! \cdot n!}\right) / \left( 2 ^ {m + n}\right)}$
        \item $\ds{\left( \frac{(k + l)!}{k! \cdot l!} \cdot \frac{((m+n) - (k + l))!}{(m - k)! \cdot (n - l)!}\right) / \left( 2 ^ {m + n}\right)}$.
    \end{itemize}

    \item Из колоды в 52 карты вынимают последовательно три карты. Найти вероятность того, что вторая из них меньше по достоинству, чем первая, а третья имеет тоже достоинство, что и вторая ($A$).

    \solution{}

    Количество вариантов вынуть последовательно три карты: $card(\Omega) = C_{52}^1 \cdot C_{51}^1 \cdot C_{50}^1$.

    Выбираем масть у первой карты. Далее выбираем достоинство для второй карты из тех достоинств, что меньше первого достоинства. Далее выбираем масть второй карты. Далее выбираем масть третьей карты, равной по достоинству второй карте: $card(A) = \sum\limits_{i = 2}^{13}  C_4^1 \cdot C_{i - 1}^1 \cdot C_4^1 \cdot C_3^1$.

    Таким образом: $\ds{P(A) = \frac{card(A)}{card(\Omega)} = \frac{\sum\limits_{i = 2}^{13}  (C_4^1) ^ 2 \cdot C_{i - 1}^1 \cdot C_3^1}{C_{52}^1 \cdot C_{51}^1 \cdot C_{50}^1}}$.
    
    \answer{} $\ds{\frac{\sum\limits_{i = 2}^{13}  (C_4^1) ^ 2 \cdot C_{i - 1}^1 \cdot C_3^1}{C_{52}^1 \cdot C_{51}^1 \cdot C_{50}^1}}$.

    \item Колода в $52$ карты раскладывается в круг. Найти вероятность того, что:
    \begin{itemize}
        \item напротив тузов лежат дамы, причем в паре туз–дама карты имеют разную масть ($A_1$);
        \item красные карты расположены против черных ($A_2$);
        \item против любой дамы — черва ($A_3$).
    \end{itemize}

    \solution{}

    Количество всех возможных разложений $52$ карт в круг: $card(\Omega) = 51!$.
    
    Первая подзадача:

    Зафиксируем Туза Черви. Далее у нас есть три варианта Дам для этого туза. Для следующего Туза остаётся $50$ мест и три варианта Дам напротив. Для следующего Туза $48$ мест и одна Дама. Для следующего $46$ мест и одна Дама. Дальше переставляем оставшиеся карты: $card(A_1) = 3 * 50 * 3 * 48 * 46 * 44!$.

    Таким образом: $\ds{P(A_1) = \frac{card(A_1)}{card(\Omega)} = \frac{3 \cdot 50 \cdot 3 \cdot 48 \cdot 46 \cdot 44!}{51!}}$.

    Вторая подзадача:

    Зафиксируем Туза Черви. Далее для следующих $25$ ячеек мы имеем два варианта их цвета: красный или чёрный (остальные ячейки определятся автоматически). Определим цвета ячеек. Далее переставим все красные карты (кроме Туза Черви), и все чёрные карты по своим ячейкам: $card(A_2) = 2^{25} \cdot 25! \cdot 26!$.

    Таким образом: $\ds{P(A_2) = \frac{card(A_2)}{card(\Omega)} = \frac{2^{25} \cdot 25! \cdot 26!}{51!}}$.

    Третья подзадача:

    Зафиксируем Даму Черви. Для неё есть $12$ вариантов расположения Червовой карты напротив. Для следующей Дамы есть $50$ мест и $11$ вариантов расположения Червовой карты напротив, и тд. Далее переставляем оставшиеся карты: $card(A_3) = 12 \cdot 50 \cdot 11 \cdot 48 \cdot 10 \cdot 46 \cdot 9 \cdot 44!$

    Таким образом: $\ds{P(A_3) = \frac{card(A_3)}{card(\Omega)} = \frac{12 \cdot 50 \cdot 11 \cdot 48 \cdot 10 \cdot 46 \cdot 9 \cdot 44!}{51!}}$.

    \answer

    \begin{itemize}
        \item $\ds{\frac{3 \cdot 50 \cdot 3 \cdot 48 \cdot 46 \cdot 44!}{51!}}$
        \item $\ds{\frac{2^{25} \cdot 25! \cdot 26!}{51!}}$
        \item $\ds{\frac{12 \cdot 50 \cdot 11 \cdot 48 \cdot 10 \cdot 46 \cdot 9 \cdot 44!}{51!}}$.
    \end{itemize}

    \item Задача про фигуры в покере без джокеров. Из колоды в $52$ карты достают $5$ карт. Найти вероятность следующих фигур из покера. А именно, найти вероятность того. что среди этих пяти карт есть:

    \begin{itemize}
        \item $3$ карты одинакового достоинства плюс $2$ одинакового (но другого) достоинства ($A_1$);
        \item $3$ одинакового достоинства и две карты, у которых достоинства отличаются друг от друга и от достоинства $3$-х карт ($A_2$);
        \item $2$ одинакового достоинства и $2$ одинакового (но другого) достоинства. Достоинство $5$-й карты отличается от достоинств $4$-х карт ($A_3$).
    \end{itemize}

    \solution{}

    Всего вариантов разложить фигуры в покере: $card(\Omega) = C_{52}^4$.

    Первая подзадача:

    Выберем первое достоинство. Выберем $3$ карты для него. Выберем второе достоинство. Выберем $2$ карты для него: $card(A_1) = 13 \cdot C_4^3 \cdot 12 \cdot C_4^2$.

    Таким образом: $\ds{P(A_1) = \frac{card(A_1)}{card(\Omega)} = \frac{13 \cdot C_4^3 \cdot 12 \cdot C_4^2}{C_{52}^4}}$. 

    Втоорая подзадача:

    Выберем первое достоинство. Выберем $3$ карты для него. Выберем $2$ следующих достоинства. И по одной картей среди каждого из них: $card(A_2) = 13 \cdot C_4^3 \cdot C_4^2 \cdot (C_4^1) ^ 2$.

    Таким образом: $\ds{P(A_2) = \frac{card(A_2)}{card(\Omega)} = \frac{13 \cdot C_4^3 \cdot C_4^2 \cdot (C_4^1) ^ 2}{C_{52}^4}}$. 

    Третья подзадача:

    Выберем два достоинства без учёта порядка (тк в обоих достоинствах выбираем по две карты). Далее выберем по две карты для каждого достоинства и доберём оставшуюся карту: $card(A_3) = C_{13}^2 \cdot (C_4^2) ^ 2 \cdot 44$.

    Таким образом: $\ds{P(A_3) = \frac{card(A_3)}{card(\Omega)} = \frac{C_{13}^2 \cdot (C_4^2) ^ 2 \cdot 44}{C_{52}^4}}$.

    \answer{}

    \begin{itemize}
        \item $\ds{\frac{13 \cdot C_4^3 \cdot 12 \cdot C_4^2}{C_{52}^4}}$
        \item $\ds{\frac{13 \cdot C_4^3 \cdot C_4^2 \cdot (C_4^1) ^ 2}{C_{52}^4}}$
        \item $\ds{\frac{C_{13}^2 \cdot (C_4^2) ^ 2 \cdot 44}{C_{52}^4}}$.
    \end{itemize}

    \item Колода $52$ карты раскладывается на $2$ не обязательно равные части. Найти вероятность того, что все черви попали в одну из кучек ($A$).

    \solution{}

    Количество вариантов разложить колоду из $52$ карт на две не обязательно равные кучки: $card(\Omega) = 2 ^ {52}$.

    Расписываем вариант, когда все черви попадут в первую кучку. Далее умножаем на $2$, так как кучки две: $card(A) = 2 \cdot \sum\limits_{i = 13}^{52} C_{39}^{i - 13}$.

    Таким образом: $\ds{P(A) = \frac{card(A)}{card(\Omega)} = \frac{2 \cdot \sum\limits_{i = 13}^{52} C_{39}^{i - 13}}{2 ^ {52}}}$.
    
    \answer{} $\ds{\frac{2 \cdot \sum\limits_{i = 13}^{52} C_{39}^{i - 13}}{2 ^ {52}}}$.
    
    \item Колода $52$ карты раскладывается в $4$ ряда по $13$ карт в каждом. Найти вероятности следующих событий:
    \begin{itemize}
        \item Все черви~--- ровно в $2$-х рядах ($A_1$);
        \item В каждом ряду есть черви ($A_2$).
    \end{itemize}
    
    \solution{}
    
    Всего вариантов разложить $52$ карты в $4$ ряда: $card(\Omega) = 52!$.
    
    Первая подзадача:
    
    Выберем два ряда для всех Червовых карт. Расставим места для Червовых карт в двух рядах. Далее исключим $4$ варианта, когда все Червовые карты будут распологаться только в одном из рядов. Переставим Червовые карты. Переставим оставшиеся карты: $card(A_1) = (C_4^2 \cdot C_{26}^{13} - 4) \cdot 13! \cdot 39!$.

    Таким образом: $\ds{P(A_1) = \frac{card(A_1)}{card(\Omega)} = \frac{(C_4^2 \cdot C_{26}^{13} - 4) \cdot 13! \cdot 39!}{52!}}$.

    Вторая подзадача:

    Посчитаем варианты разложить все Червовые карты как угодно и отнимем количество вариантов, когда все Червовые карты лежат в $3$-х, $2$-х или $1$-ом рядах: $card(A_2) = (C_{52}^{13} - C_4^3 \cdot C_{39}^{13}) \cdot 13! \cdot 39!$.

    Таким образом: $\ds{P(A_2) = \frac{card(A_2)}{card(\Omega)} = \frac{(C_{52}^{13} - C_4^3 \cdot C_{39}^{13}) \cdot 13! \cdot 39!}{52!}}$.

    \answer{}

    \begin{itemize}
        \item $\ds{\frac{(C_4^2 \cdot C_{26}^{13} - 4) \cdot 13! \cdot 39!}{52!}}$
        \item $\ds{\frac{(C_{52}^{13} - C_4^3 \cdot C_{39}^{13}) \cdot 13! \cdot 39!}{52!}}$.
    \end{itemize}

    \item Из колоды $52$ карты достают $6$ карт и раскладывают их в ряд. Событие: среди $6$ карт ровно $2$ короля и они лежат рядом.

    \solution{}

    Всего вариантов разложить $6$ карт в ряд из колоды $52$ карт: $card(\Omega) = C_{52}^6 \cdot 6!$.

    Выбираем двух королей. Выбираем $4$ карты из оставшихся некоролей. Склеиваем двух королей в одну большую карту. Домножаем на количество перестановок из $5$ элементов. Учитываем, что порядок королей в одной большой карте важен, поэтому умножаем на $2$: $card(A) = C_4^2 \cdot C_{48}^{4} \cdot 5! \cdot 2$.

    Таким образом: $\ds{P(A) = \frac{card(A)}{card(\Omega)} = \frac{C_4^2 \cdot C_{48}^{4} \cdot 5! \cdot 2}{C_{52}^6 \cdot 6!}}$.

    \answer{} $\ds{\frac{C_4^2 \cdot C_{48}^{4} \cdot 5! \cdot 2}{C_{52}^6 \cdot 6!}}$.

    \item $n$ раз бросают игральную кость. Обозначим $X_1, \ldots, X_n$ результаты бросаний, а $min X_i = X_{[1]} \leq X_{[2]} \leq \ldots \leq X_{[n]} = max X_i$ — результат упорядочивания $X_1, \ldots, X_n$ по возрастанию. Например: для $X_1 = 5, X_2 = 3, X_3 = 2, X_4 = 2 \; \rightarrow \; X_{[1]} = X_{[2]} = 2, X_{[3]} = 3, X_{[4]} = 5$. Найти вероятность того, что $X_{[k]} < 4, 1 \leq k \leq n$.

    \solution{}

    Всего вариантов бросания игральной кости $6$ раз: $card(\Omega) = 6 ^ n$.

    Выбираем места для карт, которые меньше $4$. Далее считаем количество вариантов выкинуть число меньше $3$~--- $i$ раз, и число больше $3$~--- $(n − i)$ раз. Суммируем по количеству бросков, в которых число оказалось меньше $4$: $card(A) = \sum\limits_{i = k}^n C_n^i \cdot 3 ^ n$.

    Таким образом: $\ds{P(A) = \frac{card(A)}{card(\Omega)} = \frac{\sum\limits_{i = k}^n C_n^i \cdot 3 ^ n}{6 ^ n}}$.

    \answer{} $\ds{\frac{\sum\limits_{i = k}^n C_n^i \cdot 3 ^ n}{6 ^ n}}$.

    \item В случайной перестановке чисел $1, 2, \ldots, n$ оказалось, что на $k$-ом месте стоит число, большее всех предыдущих. Найти вероятность того, что оно равно $n$ $(1 \leq k \leq n)$.

    \solution{}

    Пусть $A$~--- событие, когда на $k$-ом месте стоит $n$. А $B$~--- событие, когда на $k$-ом месте стоит число, большее всех предыдущих.

    $\ds{P(A) = \frac{(n - 1)!}{n!}} = \frac{1}{n - 1}$.

    $\ds{P(B) = \frac{\sum\limits_{i = k}^{n} \left( C_{i - 1}^{k - 1} \cdot (n - k)! \cdot (k - 1)! \right)}{n!}} = \frac{1}{k}$.

    Таким образом: $\ds{P(A|B) = \frac{P(A)}{P(B)} = \frac{1}{n - 1} : \frac{1}{k} = \frac{k}{n - 1}}$.

    \answer{} $\ds{\frac{k}{n - 1}}$.

    \item Из колоды $52$ карты достают $5$ карт, а потом ещё $5$ карт. Найти вероятность того, что во второй пятерке нет червей.

    \solution{}

    Пусть $\{B_i\}$~--- полная группа событий, где $B_i$~--- количество червовых карт в первой кучке. Событие $A$~--- во второй кучке нет червей. Выберем $i$ червей в первую кучку. Доберём оставшиеся карты. Дальше выберем не черви во вторую кучку. Распишем по формуле полной вероятности: 

    $\ds{P(A) = \sum\limits_{i = 0}^5 P(B_i) \cdot P(A|B_i) = \sum\limits_{i = 0}^5 \frac{C_{13}{i} \cdot C_{39}^{5 - i}}{C_{52}^5} \cdot \frac{C_{34 + i}^5}{C_{47} ^ 5}}$.

    \answer{} $\ds{  \sum\limits_{i = 0}^5 \frac{C_{13}{i} \cdot C_{39}^{5 - i}}{C_{52}^5} \cdot \frac{C_{34 + i}^5}{C_{47} ^ 5}}$.

    \item Из колоды $52$ карты достают $5$ карт. Обозначим $k$ число тузов среди этих карт. После этого их другой колоды в $52$ карт достают $(k + 1)$ карту. Найти вероятность того, что среди них:
    \begin{itemize}
        \item нет королей;
        \item нет пик;
    \end{itemize}

    \solution{}

    Первая подзадача:

    Пусть ${B_i}$~--- полная группа событий, где $B_i$~--- количество тузов в первой кучке. Событие $A$~--- во второй кучке нет королей. Выберем $i$ тузов в первую кучку. Доберём оставшиеся карты. Дальше выберем во вторую кучку $(i + 1)$ не королей. Распишем по формуле полной вероятности:

    $\ds{P(A) = \sum\limits_{i = 0}^4 P(B_i) \cdot P(A|B_i) = \sum\limits_{i = 0}^4 \frac{C_4^i \cdot C_{48}^{5 - i}}{C_{52}^5} \cdot \frac{C_{48}^{i + 1}}{C_{52}^{i + 1}}}$.

    Вторая подзадача:

    Пусть теперь событие $A$~--- во второй кучке нет королей. Выберем $i$ тузов в первую кучку. Доберём оставшиеся карты. Дальше выберем во вторую кучку $(i + 1)$ не пик. Распишем по формуле полной вероятности:

    $\ds{P(A) = \sum\limits_{i = 0}^4 P(B_i) \cdot P(A|B_i) = \sum\limits_{i = 0}^4 \frac{C_4^i \cdot C_{48}^{5 - i}}{C_{52}^5} \cdot \frac{C_{39}^{i + 1}}{C_{52}^{i + 1}}}$.

    \answer{}

    \begin{itemize}
        \item $\ds{\sum\limits_{i = 0}^4 \frac{C_4^i \cdot C_{48}^{5 - i}}{C_{52}^5} \cdot \frac{C_{48}^{i + 1}}{C_{52}^{i + 1}}}$.
        \item $\ds{\sum\limits_{i = 0}^4 \frac{C_4^i \cdot C_{48}^{5 - i}}{C_{52}^5} \cdot \frac{C_{39}^{i + 1}}{C_{52}^{i + 1}}}$..
    \end{itemize}

    \item Найти вероятность того, что в случайной перестановке чисел $1, 2, \ldots, n$ все четные числа расположены в порядке возрастания, если известно, что все нечетные расположены в порядке убывания. 

    \solution{}

    $A$~--- множество нечетных чисел в порядке убывания.

    $B$~--- множество чётных чисел в порядке возрастания.

    Выбрали места для чётных чисел: $\ds{P(A \cap B) = \frac{C_n^{\lfloor n / 2 \rfloor}}{n!}}$.

    Выбрали места для чётных чисел и отсортировали их: $\ds{P(A) = \frac{C_n^{\lfloor n / 2 \rfloor} \cdot \lfloor\frac{n}{2}\rfloor !}{n!}}$.

    Выбрали места для чётных чисел и отсортировали нечётные: $\ds{P(B) = \frac{C_n^{\lfloor n / 2 \rfloor} \cdot \lfloor\frac{n}{2}\rfloor !}{n!}}$.

    $\ds{P(A) \cdot P(B) = \frac{C_n^{\lfloor n / 2 \rfloor} \cdot \lfloor\frac{n}{2}\rfloor ! \cdot C_n^{\lfloor n / 2 \rfloor} \cdot \lfloor\frac{n}{2}\rfloor !}{(n!) ^ 2}} = \frac{C_n^{\lfloor n / 2 \rfloor}}{n!} = P(A \cap B)$.

    $\ds{P(A|B) = P(A) = \frac{C_n^{\lfloor n / 2 \rfloor} \cdot \lfloor\frac{n}{2}\rfloor !}{n!}}$.

    \answer{} $\ds{\frac{C_n^{\lfloor n / 2 \rfloor} \cdot \lfloor\frac{n}{2}\rfloor !}{n!}}$.

    \item Каждое воскресенье мальчик играет во дворе в настольный теннис. Ему предлагают сыграть три партии на выбор по одной из трех схем: либо сначала сыграть с чемпионом двора, потом~--- со своим отцом и третью партию снова с чемпионом, либо в обратном порядке~--- с отцом, потом~--- с чемпионом и, наконец, снова с отцом. Какой порядок более выгоден мальчику, если
    \begin{itemize}
        \item ему нужно выиграть хотя бы две партии подряд;
        \item ему нужно выиграть хотя бы две парти из трёх.
    \end{itemize}
    Ни отец мальчика, ни сам мальчик не являются чемпионом двора.

    \solution{}

    Первая подзадача:

    Пусть $A_1$~--- выигрыш по первой схеме, $A_2$~--- по второй схеме, $B_1$~--- выигрыш у отца, $B_2$~--- выигрыш у чемпиона.

    $P(A_1) = P(B_2) \cdot P(B_1) \cdot (1 - P(B_2)) + P(B_2) \cdot P(B_1) \cdot P(B_2) + (1 - P(B_2)) \cdot P(B_1) \cdot P(B_2) = P(B_1) \cdot P(B_2) + (1 - P(B_2)) \cdot P(B_1) \cdot P(B_2)$.

    $P(A_2) = P(B_1) \cdot P(B_2) \cdot (1 - P(B_1)) + P(B_1) \cdot P(B_2) \cdot P(B_1) + (1 - P(B_1)) \cdot P(B_2) \cdot P(B_1) = P(B_1) \cdot P(B_2) + (1 - P(B_1)) \cdot P(B_2) \cdot P(B_1)$.

    $P(A_1) - P(A_2) = P(B_1) - P(B_2) > 0$ \rightarrow $A_1$ выгоднее.

    Вторая подзадача:

    $P(A_1) = P(B_2) \cdot P(B_1) \cdot (1 - P(B_2)) + P(B_2) \cdot P(B_1) \cdot P(B_2) + (1 - P(B_2)) \cdot P(B_1) \cdot P(B_2) + P(B_2) \cdot (1 - P(B_1)) \cdot P(B_2)$.

    $P(A_2) = P(B_1) \cdot P(B_2) \cdot (1 - P(B_1)) + P(B_1) \cdot P(B_2) \cdot P(B_1) + (1 - P(B_1)) \cdot P(B_2) \cdot P(B_1) + P(B_1) \cdot (1 - P(B_2)) \cdot P(B_1)$.

    $P(A_1) - P(A_2) = -2 \cdot P(B_1) \cdot P(B_2) + P(B_2) - P(B_1) + 2 \cdot P(B_2) \cdot P(B_1) = (P(B_2) - P(B_1)) \cdot (P(B_1) + P(B_2) - 2 P(B_1) \cdot P(B_2)) < 0$ \rightarrow $A_2$ выгоднее.

    \answer{}

    \begin{itemize}
        \item $A_1$ выгоднее
        \item $A_2$ выгоднее.
    \end{itemize}


    \item Двое дуэлянтов стреляются на следуюших условиях: сначала стреляет первый дуэлянт, если он попадает, то дуэль заканчивается в его пользу, если же он промахивается, то стреляет второй. Если он попал, то дуэль кончилась (второй победил), нет~--- снова стреляет первый и так далее до первого попадания любого из стреляющихся (запасы пороха не ограничены). Пусть первый дуэлянт попадает в цельс  вероятностью $p_1$, а второй~--- с вероятностью $p_2$. Чему равна вероятность того, что 
    \begin{itemize}
        \item дуэль закончится в пользу первого стрелка?
        \item в пользу второго стрелка?
        \item ничья?
        \item При каком соотношении $p_1$ и $p_2$ правила дуэли одинаково справедливы для обоих дуэлянтов?
    \end{itemize}

    \solution{}

    Первая подзадача: $\ds{P_1 = p_1 + (1 - p_1) \cdot (1 - p_2) \cdot p_1 + \ldots = \frac{p_1}{1 - (1 - p_1)\cdot(1 - p_2)}}$ (по формуле геометрической прогрессии).

    Вторая подзадача: $\ds{P_2 = (1 - p_1) \cdot p_2 + (1 - p_1) \cdot (1 - p_2) \cdot (1 - p_1) \cdot p_2 + \ldots = \frac{(1 - p_1) \cdot p_2}{1 - (1 - p_1) \cdot (1 - p_2)}}$ (по формуле геометрической прогрессии).

    Третья подзадача: $\ds{P_3 = \lim\limits_{n \rightarrow \infty}(1 - p_1)^n \cdot (1 - p_2) ^ n} = 0$

    Четвертая подзадача: $\ds{P_1 = P_2 \; \Rightarrow \; \frac{p_1}{1 - (1 - p_1)\cdot(1 - p_2)} = \frac{(1 - p_1) \cdot p_2}{1 - (1 - p_1) \cdot (1 - p_2)} \Rightarrow p_2 = \frac{p_1}{1 - p_1}}$.

    \answer{}
    \begin{itemize}
        \item $\ds{ \frac{p_1}{1 - (1 - p_1)\cdot(1 - p_2)}}$
        \item $\ds{ \frac{(1 - p_1) \cdot p_2}{1 - (1 - p_1) \cdot (1 - p_2)}}$
        \item $\ds{0}$
        \item $\ds{ p_2 = \frac{p_1}{1 - p_1}}$
    \end{itemize}
    
\end{enumerate}

\section*{Прошлые наработки по задачам из СУНЦа, удалить после завершения}
























 
	\task{$\ds \sin{\left(\frac{\pi}{6}\cos{2x}\right)}=\cos{\left(\frac{4\pi}{3}\sin{x}\right)}$}{$\frac{\pi}{2}+\pi k; \pm\arcsin{\left(2-\sqrt{3}\right)}+\pi n, n \in \mathds{Z}, k \in \mathds{Z}$}
	
	\task{$\ds \sin{\left(\pi\sqrt{x}\right)}=\cos{\left(\pi\sqrt{2-x}\right)}$}{$1 + \frac{\sqrt{15}}{8}$}


\section*{Системы тригонометрических уравнений}
\begin{enumerate}
	\task{$\ds
		\left \{ 
			\begin{array}{ll}
 				\tg{y}-\tg{x}=1+\tg{x}\cdot\tg{y}, \\
 				\cos{2y}+\sqrt{3}\cos{2x} = -1 \\
  			\end{array}
		\right.
	$}
	{$\left(\frac{7}{12}\pi+\pi f; \frac{5}{6}\pi+\pi \left(k+f \right)\right)\!, \,k, f \in \mathds{Z}$}
	
	\task{$\ds
		\left \{
			\begin{array}{lll}
 				\cos{x} \geq 0, \\
 				\cos{x}\cdot\sin{y}=\frac{\sqrt{6}}{4}, \\
 				\sin^2{x}+\cos^2{y}=\frac{3}{4} \\
  			\end{array}
  		 \right.
  	$}	
	{$
		\begin{array}{ll}
			\left \{ 
				\begin{array}{ll}
 					x = \pm \frac{\pi}{6} + 2\pi k, k \in \mathds{Z}, \\
 					y = \left(-1\right)^{n} \frac{\pi}{4} + \pi n, n \in \mathds{Z}\\
  				\end{array}
  			\right. \\
 			\\
  			\left\{ 
  				\begin{array}{ll}
 					x = \pm \frac{\pi}{4} + 2\pi f, f \in \mathds{Z}, \\
 					y = \left(-1\right)^{m} \frac{\pi}{3} + \pi m, m \in \mathds{Z}\\
  				\end{array}
  			\right.
		\end{array}
	$}
\end{enumerate}

\section*{Равенства с обратными тригонометрическими функциями}
\begin{enumerate}
	\task{$\ds \arcsin{\left(\sin{5}\right)} = x$}	{$5 - 2\pi$}
	
	\task{$\ds \arccos{\left(\cos{5}\right)} = x$}{$2\pi - 5$}
	
	\task{$\ds 2\arctg{\frac{1}{3}} + \arcsin{\frac{4}{5}} = x$}{$ \frac{\pi}{2}$}
	
	\task{$\ds \arccos{x} = \arctg{x}$}{$ \sqrt{\frac{\sqrt{5}-1}{2}}$}
	
	\task{$\ds \arcsin{\frac{x}{2}} + 2\arccos{x} = \pi$}{$0$}
	
	\task{$\ds \left(13x^2+2x-14\right)\cdot \arccos{x} = 0$}{$1; \frac{\sqrt{183}-1}{13}$}
	
	\task{$\ds \cos{\left(\frac{4+\sqrt{5}}{2}\sin{x}+2\cos{x}\right)}=\sin{\left(\frac{\sqrt{5}-4}{2}\sin{x}\right)}$}{$\arccos{\frac{2}{3}}+\arccos{\frac{\pi}{6}}; \arccos{\frac{\sqrt{5}}{5}}+\pi-\arccos{\left(\frac{\pi\sqrt{5}}{20}\right)}$}
	
	\task{$\ds \arcsin{\left(\frac{3-5\sin{\left(x+\frac{\pi}{3}\right)}-6\cos{2x}}{5}\right)}=x-\frac{\pi}{3}$}{$\arcsin{\frac{3}{4}};\pi-\arcsin{\frac{3}{4}};-\arcsin{\frac{1}{3}}$}
\end{enumerate}

\section*{Неравенства с обратными тригонометрическими функциями}
\begin{enumerate}
	\task{$\ds 2\cos{\left(\arcsin{x}\right)} - \sin{\left(\frac{\arccos{x}}{2}\right)} \leq 0$}{$x \in \left[ -1 ; -\frac{7}{8} \right] \cup \{ 1 \} $}
\end{enumerate}

\section*{Логарифмические и показательные равенства}
\begin{enumerate}
	\task{$\ds \log_{2}{3} \cdot \log_{x+5}{4} - \log_{4}{\left(x-5\right)^2} \cdot \log_{x+5}{2} = 1$}{$4; \sqrt{34}$}
	
	\task{$ \ds 2\cdot7^{\log_{2x}{\left(x^2-1\right)^2}}-9\cdot14^{\log_{2x}{\left(x^2-1\right)^2}}+7\cdot4^{\log_{2x}{\left(x^2-1\right)}} = 0$}	{$\sqrt{2}; 1 + \sqrt{2}$}
	
	\task{$\ds \log_{x+2}{\log_{2}{\log_{x+1}{\left(11x^2+12x\right)}}} = 0$}{$\frac{\sqrt{35} - 5}{10}$}
	
	\task{$\ds \log_{3}{x} + \log_{x}{\left(\frac{1}{2}\right)} = \log_{5}{x}$}{$\ds 2^{\pm\left(\sqrt{\frac{\log_{3}{2}\cdot\log_{2}{5}}{\log_{2}{\frac{5}{3}}}}\right)}$}
	
	\task{$\ds \log_{3}^2{\left(2x+1\right)} + \log_{\left(x+4\right)}^2{3} = \log_{3}^2{\left(x+4\right)} + \log_{\left(2x+1\right)}^2{3}$}{$3; \frac{-9+\sqrt{57}}{4}$}
	
	\task{$\ds 2\log_{\left(x+3\right)}{\left(12+10x+2x^2\right)} + \frac{1}{2}\log_{\left(4+2x\right)}{\left(x^2+6x+9\right)} = 5$}{$-1; \frac{-15+\sqrt{17}}{8}$}
\end{enumerate}

\section*{Системы логарифмических и показательных уравнений}
\begin{enumerate}
	\task{$\ds 
		\left \{
			\begin{array}{ll}
				x^{\log_{y}{3}} = 2\sqrt{x}\\
				y^{\log_{2}{x}} = 9
			\end{array}
		\right.
	$}
	{$\ds \big(4;3\big), \big(\frac{1}{2};\frac{1}{9}\big)$}
	
	\task{$\ds
		\left \{
			\begin{array}{ll}
				\log_{\left(x+1\right)}{\left(2x+3\right)} + \log_{\left(x+2\right)}{\left(x^2+2\right)} = 1, \\
				\log_{\left(x+2\right)}{\left(x^2+2\right)} + \log_{\left(x+1\right)}{\left(10x+6\right)} = 2
			\end{array}
		\right.
	$}
	{$-\frac{1}{2}$}
	
	\task{$\ds
		\left \{
			\begin{array}{ll}
				\log_{y}{x}\cdot\log_{x}{\left(y-x\right)} - \log_{y}{x} = 1 - 2\log_{y}{x}\cdot \log_{x}{6}, \\
				\log_{y\left(y-x\right)}{x} + \log_{\frac{x}{324}}{x} = 0
			\end{array}
		\right.
	$}
	{$\left(9;12\right)$}
\end{enumerate}

\section*{Смешанные равенства}
\begin{enumerate}
	\task{$\ds \left|\sin{3x}\right|^{\tg{5x}} = 1$}{$\frac{\pi}{6} + \frac{\pi f}{3}; \frac{\pi k}{5}, f \neq 3 r + 1, k \neq 5 s,\, f, k, s, r \in \mathds{Z}$}
	
	\task{$\ds \log_{\cos{x}}{\sin{x}} + \log_{\sin{x}}{\cos{x}} = 2$}{$\frac{\pi}{4} + 2\pi k, k \in \mathds{Z}$}
\end{enumerate}

\section*{Логарифмические и показательные неравенства}
\begin{enumerate}
	\task{$ \ds \log_{\left(\sqrt{3}-1\right)}{\left(x+20\right)^2} \leq \log_{\left(\sqrt{3} - 1\right)}{\Big(2-\sqrt{3}\Big)} \cdot \log_{\left(2-\sqrt{3}\right)}{\Big(\big(x+20\big)\big(x^2-2x-8\big)\Big)}$}	{$x \in \left[ -4; -2\right) \cup \left(4;7\right]$}

	\task{$ \ds \log_{3\sqrt{2}}{\left(x^2-6x+4\right)} + \log_{\frac{\sqrt{2}}{6}}{\left(5 - 4,5x - 0,5x^2\right)} \geq 0$}	{$x \in \left(-10; \frac{3-\sqrt{33}}{6} \right]$}

	\task{$ \ds \frac{1}{x} \cdot \log_{\frac{4}{10}}{\left(\frac{12 - 4 \cdot 5^{-x}}{5}\right)} \leq \log_{\frac{5}{2}}{\frac{1}{5}}$}	{$x \in \left(-\log_{5}{3}; \log_{5}{2}-1\right] \cup \left(0;\log_{5}{2}\right]$}

	\task{$ \ds \log_{4}{\left(\left|2x+1\right|-\left|x-2\right|\right)} \geq \log_{2}{\sqrt{\frac{x-2}{x-3}}}$}		{$x \in \left(-\infty; \frac{-7-\sqrt{21}}{2}\right] \cup \Big(2;+\infty\Big)$}

	\task{$\ds \log_{\sqrt[3]{x-1}}{\left[ \frac{\left(x-1\right)^2}{2\left(x-\frac{5}{3}\right)^2}\right]} \geq 6 $}	{$x \in \left(2; \frac{10+3\sqrt{2}}{6}\right]$}

	\task{$\ds 2\log_{\frac{1}{2}}{\left(\log_{\frac{1}{x}}{2}\right)} + 5 \log_{\sqrt{x}}{\left(\log_{\frac{1}{2}}{x}\right)} \leq 0$}	{$x \in \left[\frac{1}{32};\frac{1}{2}\right]$}

	\task{$\ds 2 + \log_{\left(x-1\right)}{\left(\frac{10}{6x^2-15}\right)} \leq 0$}	{$x \in \left(\frac{\sqrt{10}}{2};2\right) \cup \left\{\frac{5}{2}\right\}$}

	\task{$\ds 4 + \log_{4}{\left(x^2+6x+8\right)} > \frac{1}{\log_{x+2}{4}} + \frac{1}{\log_{x+4}{\sqrt{2}}}$}	{$x \in \bigg(-2;-1\bigg) \cup \bigg(-1; 4\cdot\Big(2^{\frac{2}{3}} - 1\Big)\bigg)$}

	\task{$\ds \left(2^{x+1}-3^{x+1}\right)\sqrt{2^{2x}-2^{x+3}\cdot3^{x}+11\cdot3^{2x}} \geq 0$}	{$x \in \left(-\infty;\log_{\frac{2}{3}}{\left(4+\sqrt{5}\right)}\right]\cup\left[\log_{\frac{2}{3}}{\left(4-\sqrt{5}\right)};-1\right]$}

	\task{$\ds \left(x^2+\frac{1}{2}\right)^{3x^2-10x+2} > \frac{2}{2x^2+1}$}	{$x \in \left(-\infty;-\frac{\sqrt{2}}{2}\right)\cup\left(\frac{1}{3};\frac{\sqrt{2}}{2}\right)\cup\Big(3;+\infty\Big)$}

	\task{$\ds \left|x+1\right|^{2\sqrt{x+3}} < \left|x+1\right|^{1-x}$}	{$x \in \left(-3;-2\right)\cup\left(3 - 2\sqrt{5};-1\right)\left(-1;0\right)$}

	\task{$\ds 4\sqrt{\frac{2^x-1}{2^x}} + \sqrt{14} \leq 14\sqrt{\frac{2^{x-2}}{2^x-1}}$}	{$x \in \left(0;3\right]$}

	\task{$\ds \log_{\frac{1}{2}}{\log_{\frac{137}{20}}{\left(\left(\frac{2}{5}\right)^x+\frac{3}{5}\right)}} \geq 0$}	{$x \in \left[-2;1\right)$}

	\task{$\ds \left(1 + \log_{3}{x}\right)\sqrt{\log_{3x}{\left(\sqrt[3]{\frac{x}{3}}\right)}} \leq 2$}{$x \in \left(0;\frac{1}{3}\right) \cup \left[3;3^{\sqrt{13}}\right]$}

	\task{$\ds \log_{2}{\left(5-x\right)} \cdot \log_{\left(x+1\right)}{\frac{1}{8}} \geq -6$}{$x \in \left(-1;0\right) \cup \left[1;5\right)$}
\end{enumerate}

\section*{Смешанные неравенства}
\begin{enumerate}
	\task{$\ds \log_{\left|\sin{x}\right|}{\left(x^2-8x+23\right)} > \frac{3}{\log_{2}{\left|\sin{x}\right|}}$}{$x \in \big(3; \pi \big) \cup \big(\pi; \frac{3\pi}{2}\big) \cup \big(\frac{3\pi}{2};5\big)$}
	
	\task{$\ds x^{\lg{\sin{x}}} \geq 1$, if $x > 0$}{$x \in \left(0;1\right] \cup \left\{\frac{\pi}{2} + 2\pi k \right\}, k \geq 0, k \in \mathds{Z}$}
\end{enumerate}

\end{document}
